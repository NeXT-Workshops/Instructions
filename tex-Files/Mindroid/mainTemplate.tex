%% This is file `DEMO-TUDaPub.tex' version 2.09 (2020/03/13),
%% it is part of
%% TUDa-CI -- Corporate Design for TU Darmstadt
%% ----------------------------------------------------------------------------
%%
%%  Copyright (C) 2018--2020 by Marei Peischl <marei@peitex.de>
%%
%% ============================================================================
%% This work may be distributed and/or modified under the
%% conditions of the LaTeX Project Public License, either version 1.3c
%% of this license or (at your option) any later version.
%% The latest version of this license is in
%% http://www.latex-project.org/lppl.txt
%% and version 1.3c or later is part of all distributions of LaTeX
%% version 2008/05/04 or later.
%%
%% This work has the LPPL maintenance status `maintained'.
%%
%% The Current Maintainers of this work are
%%   Marei Peischl <tuda-ci@peitex.de>
%%   Markus Lazanowski <latex@ce.tu-darmstadt.de>
%%
%% The development respository can be found at
%% https://github.com/tudace/tuda_latex_templates
%% Please use the issue tracker for feedback!
%%
%% ============================================================================
%%
% !TeX program = lualatex
%%

\documentclass[
	ngerman,
	accentcolor=1c,% Farbe für Hervorhebungen auf Basis der Deklarationen in den Corporate Design Richtlinien
%	logofile=example-image, %Falls die Logo Dateien nicht vorliegen
	marginpar=false,
	identbarcolor=1c,
	]{tudapub}

\usepackage[english, main=ngerman]{babel}
\usepackage[babel]{csquotes}

\usepackage{biblatex}
\bibliography{DEMO-TUDaBibliography}

%Formatierungen für Beispiele in diesem Dokument. Im Allgemeinen nicht notwendig!
\let\file\texttt
\let\code\texttt
\let\pck\textsf
\let\cls\textsf





%% LOAD CONFIGURATION

% USEPACKAGE
% changed counter for section wise counting
\usepackage{chngcntr}
%\usepackage[utf8]{inputenc}
\usepackage[T1]{fontenc}
%\usepackage[ngerman]{babel}
\usepackage{graphicx}
\usepackage{subcaption}
\usepackage{listings}
\setlength {\marginparwidth }{2cm} 
\usepackage{todonotes}
%\usepackage[hyphens]{url}
\usepackage{ifthen}
\usepackage{array,multirow,colortbl}

\newboolean{withJava}
\newboolean{doc}
\newboolean{devDoc}
\newboolean{solution}
\newboolean{tasks}
\newboolean{intro}
\newboolean{scoring}

\counterwithin{figure}{section} 
\counterwithin{table}{section} 

\newcommand{\gcenter}[4]{
	\begin{figure}[h]
	\centering 
	\includegraphics[width=#4]{#1}
	\caption{#2}
	\label{fig:#3}
	\end{figure}
}

\newcommand{\easygcenter}[2]{
	\begin{figure}[h]
	\centering 
	\includegraphics[width=#1]{#2}
	\end{figure}
}

\newcommand{\gcenterone}[1]{
	\begin{figure}[h]
	\centering 
	\includegraphics[width=.9\textwidth]{#1}
	\end{figure}
}

\newcommand{\td}[1]{
	\todo[inline]{#1}
	}
	
\newcommand{\sol}{\begingroup
	\catcode`_=12 \docodelst}

\newcommand{\docodelst}[1]{
	\lstinputlisting[caption=\texttt{#1.java}]{\solpath/#1.java}
	\endgroup
}

%makes bold code snippets
\newcommand{\bfcode}[1]{\texttt{\textbf{#1}}}

%fixes enum labels
\renewcommand{\labelenumi}{\theenumi .}
\renewcommand{\labelenumii}{\theenumii )}

%define colors for listings		
\definecolor{javared}{rgb}{0.6,0,0} 			% for strings
\definecolor{javagreen}{rgb}{0.25,0.5,0.35}    	% comments
\definecolor{javapurple}{rgb}{0.5,0,0.35} 		% keywords
\definecolor{javadocblue}{rgb}{0.25,0.35,0.75}  % javadoc

%configure listings for Java Code 
\lstset{
	language=Java,
	breaklines=true,
	postbreak=\mbox{$\hookrightarrow$\space},
	basicstyle=\ttfamily\footnotesize,
	%set colors
	keywordstyle=\color{javapurple}\bfseries,
	stringstyle=\color{javared},
	commentstyle=\color{javagreen},
	morecomment=[s][\color{javadocblue}]{/**}{*/},
	%set line numbering
	numbers=left,
	numberstyle=\tiny\color{black},
	%stepnumber=2, %is uncommented numers every 2nd line
	numbersep=10pt,
	tabsize=4,
	showspaces=false,
	showstringspaces=false
}

%% MANAGE BOOLEANS

% DOCU

\setboolean{withJava}{true}
\setboolean{doc}{true}
\setboolean{devDoc}{false}

% TASKS
\setboolean{tasks}{false}
\setboolean{solution}{false}
\setboolean{intro}{true}
\setboolean{scoring}{false}

%relative path to workshop solutions folder
\newcommand{\solpath}{solutions}

\newcommand{\mySecName}{DUMMY}

\ifthenelse{\boolean{devDoc}}
{\renewcommand{\mySecName}{Abschnitt}}
{\renewcommand{\mySecName}{Aufgabe}}
\ifthenelse{\boolean{doc}}
{\renewcommand{\mySecName}{Abschnitt}}
{\renewcommand{\mySecName}{Aufgabe}}

\makeatletter
\newcommand{\TUD@sectionname@de}{\mySecName}
\makeatother






\begin{document}

%Zusätzliche Metadaten für PDF/A. In diesem Fall notwendig, weil Titel ein Makro enthält.
\Metadata{
	author=NeXT,
	title=Mindroid Template,
	%subject=Basisdokumentation und Template zur Nutzung der tudapub-Dokumentenkasse,
	%date=2019-04-29,
	%keywords=TU Darmstadt \sep Corporate Design \sep LaTeX
}




\title{Space Workshop\newline Dokumentation}
\subtitle{NeXT Generation on Campus}
%\author{Marei Peischl\thanks{pei\TeX{} \TeX{}nical Solutions}\and der \TeX-Löwe}
\date{}
%\titleimage{
%%	%Folgende Box kann selbstverständlich durch ein mit \includegraphics geladenes Bild ersetzt werden.
%\bigskip
%\bigskip
%\bigskip
%\bigskip
%\bigskip
%\bigskip
%\bigskip
%\bigskip
%\bigskip
%\bigskip
%\bigskip
%%\includegraphics[width=\textwidth]{images/title.jpg}
%	%\color{black!30}\rule{\width}{\height}
%}


%Varianten der Infoboxen
%\addTitleBox{\includegraphics[width=\linewidth]{images/ist_logo.pdf}}
%\addTitleBoxLogo{example-image}
%\addTitleBoxLogo*{\includegraphics[width=.3\linewidth]{example-image}}



%\maketitle


%\newpage


	\titleimage{
	\centering
	%	%Folgende Box kann selbstverständlich durch ein mit \includegraphics geladenes Bild ersetzt werden.
	\bigskip
	\bigskip
	\bigskip
	\bigskip
	\bigskip
	\bigskip
	\bigskip
	\bigskip
	\bigskip
	\bigskip
%	\bigskip	
	\includegraphics[width=0.8\textwidth]{MindroidTitle.png}
	%\color{black!30}\rule{\width}{\height}
}


%Varianten der Infoboxen
\addTitleBox{\includegraphics[width=\linewidth]{../ist_logo.pdf}}
%\addTitleBoxLogo{example-image}
%\addTitleBoxLogo*{\includegraphics[width=.3\linewidth]{example-image}}
	
	
	
	\newcommand{\titletext}{No Title - No Content}
	
	\ifthenelse{\boolean{doc}}{
		\renewcommand{\titletext}{Dokumentation}
	}{}
	\ifthenelse{\boolean{devDoc}}{
		\renewcommand{\titletext}{Entwickler-Dokumentation}
	}{}
	
	\ifthenelse{\boolean{tasks}}{
		\ifthenelse{\boolean{solution}}{
			\renewcommand{\titletext}{Aufgabenstellung \\mit Lösungen}
		}{
			\renewcommand{\titletext}{Aufgabenstellung}
		}
	}{
		\ifthenelse{\boolean{solution}}{
			\renewcommand{\titletext}{Lösungen}
		}{}
	}


	\pagenumbering{arabic}	
	\title{Mindroid Workshop \\ \titletext}
	\subtitle{NeXT Generation on Campus\newline TU Darmstadt}
%	\subsubtitle{}
	
	\maketitle	
	
	\ifthenelse{\boolean{solution}}{
		\centering	
		\bigskip\bigskip\bigskip\bigskip\bigskip\bigskip\bigskip\bigskip\bigskip	
		{\Huge LÖSUNGEN}
		\newpage
	}{	
%	\bigskip\bigskip\bigskip\bigskip\bigskip
%		\easygcenter{.9\textwidth}{logo/MindroidTitle.png}
%		\newpage
}
	
	
	\ifthenelse{\boolean{doc}}{
			
	
	\section{Einführung}
	In dieser Übersicht werden die Funktionen, die zur Steuerung der Roboter zur Verfügung stehen erklärt.
	Zur Verdeutlichung ein kleines Beispiel:
	
		\begin{table}[htbp]
		\begin{tabular}{|p{0.2\textwidth} p{0.75\textwidth}|}
			\hline
			\textbf{Typ} & \textbf{Methode und Beschreibung} \\ \hline
			void & delay(long milliseconds) \\ 
			&\\
			& Verzögert die Ausführung um die angegebene Zeitspanne (Milisekunden)\\ \hline
		\end{tabular}
		\end{table}
	
	Die Spalte \textit{Typ }gibt an, welchen Typ der Rückgabewert der Funktion hat. \textit{void} bedeutet, dass kein Wert zurück gegeben wird. 
	In der Klammer hinter dem Funktionsnamen wird angegeben, welche Parameter die Funktion erwartet, und von welchem Typ diese sein müssen. In unserem Beispiel bedeutet dies, dass die \textit{delay}-Methode einen Parameter vom Typ \textit{long} (ganzzahliger Wert) erwartet, welcher \textit{milliseconds }genannt wird. 
	Ein möglicher Funktionsaufruf sieht wie folgt aus:
	
	\begin{lstlisting}
	public void run(){	
		delay(1000);
	}
	\end{lstlisting}
	
	Dabei wird die delay-Methode mit 1000 als Parameter aufgerufen. Das bedeutet, die Ausführung wird um $1000ms$ ($= 1s$) verzögert.
	
	
	\subsection{isInterrputed}
	Damit die Ausführung des Programms auch in Schleifen unterbrochen werden kann, sollte jede Schleife die isInterrupted-Methode abfragen. 
	
	Beispiel:
	\begin{lstlisting}
		public void run(){	
			while(!isInterrupted()){
				// Schleifeninhalt
			}
			for(int i=0; i<10 && !isInterrupted(); i++){
				// Schleifeninhalt
			}
		}
	\end{lstlisting}
	
	

	\section{Wichtige Funktionen}
	Hier eine kleine Übersicht über die wichtigsten Funktionen beim Programmieren der Roboter.
	
	\subsection{Fahren}
		\begin{center}\texttt{import org.mindroid.api.ImperativeWorkshopAPI}\end{center}
		
		Mögliche Eingabewerte für den $speed$-Parameter liegen zwischen 0 und 1000.
		Eine maximale Geschwindigkeit von $300$ sollte ausreichen. Niedrigere Geschwindigkeiten schonen den Akku.	
		Die Distanz wird im $distance$-Parameter immer als Kommazahl in Zentimetern (cm) angegeben (z.B.: 20cm werden als $20.0f$ angegeben)
	
		\begin{table}[htbp]
		\begin{tabular}{|p{0.2\textwidth} p{0.75\textwidth}|}
		\hline
		\textbf{Typ} & \textbf{Methode und Beschreibung} \\ \hline
		void & setMotorSpeed(int speed) \\
		& \\
		& Bestimmt die Geschwindigkeit für Fahrmethoden ohne $speed$-Parameter. \\ \hline
		void & forward() \\ 
		void & backward() \\ & \\
		& Fahren mit der von $setMotorSpeed(...)$ gesetzten Geschwindigkeit. \\ \hline
		void & driveDistanceForward(float distance) \\
		void & driveDistanceBackward(float distance) \\ & \\
		& Fahren mit der von $setMotorSpeed(...)$ gesetzten Geschwindigkeit\\ 
		& Die Distanz muss in Zentimetern angegeben werden. \\ \hline
		
		void & forward(int speed) \\ 
		void & backward(int speed)  \\ 
		void & driveDistanceForward(float distance, int speed) \\ 
		void & driveDistanceBackward(float distance, int speed) \\ 		
		& \\
		& Wie oben, nur dass der $speed$-Parameter die von $setMotorSpeed()$ gesetzte Geschwindigkeit überschreibt. Nach Beendigung des Aufrufs, wird wieder die vorher gesetzte Geschwindigkeit genutzt.\\ \hline

		void & turnLeft(int degrees) \\ 
		void & turnRight(int degrees) \\ 
		void & turnLeft(int degrees, int speed) \\ 
		void & turnRight(int degrees, int speed) \\ 
		& \\
		& Dreht den Roboter um den im $degrees$-Parameter bestimmten Wert. \\
		& Der $Speed$-Parameter verhält sich wie bei den anderen Methoden. \\ \hline
		
		void & stop() \\ 
		& \\
		& Stoppt sofort alle Motoren.	\\ \hline
		\end{tabular}
		\end{table}
		
		\newpage
	\subsection{Sensoren}
			\begin{center}\texttt{import org.mindroid.api.ImperativeWorkshopAPI}\end{center}
		\begin{table}[htbp]
		\begin{tabular}{|p{0.2\textwidth} p{0.75\textwidth}|}
		\hline
		\textbf{Typ} & \textbf{Methode und Beschreibung} \\ \hline
		float & getAngle() \\ 
		&\\
		& Liefert den Winkel des Gyrosensors in Grad\\ \hline
		float & getDistance() \\ 
		&\\
		& Liefert die vom Ultraschallsensor gemessene Distanz in Zentimetern\\ \hline		
		Colors & getLeftColor() \\ 
		Colors & getRightColor() \\
		&\\
		& Liefert den Wert des Linken/Rechten Farbsensors\\ 
		& Farbwerte: Colors.BLACK, Colors.BLUE, Colors.BROWN, Colors.GREEN, \\ &Colors.RED, Colors.WHITE, Colors.YELLOW, Colors.NONE\\ \hline
		\end{tabular}
		\end{table}

	\subsection{Kommunikation}
				\begin{center}\texttt{import org.mindroid.api.ImperativeWorkshopAPI}\end{center}
		\begin{table}[htbp]
		\begin{tabular}{|p{0.2\textwidth} p{0.75\textwidth}|}
		\hline
		\textbf{Typ} & \textbf{Methode und Beschreibung} \\ \hline
		boolean & hasMessage() \\ 
				&\\
		& Prüft ob Nachricht vorhanden ist \\ \hline
		MindroidMessage & getNextMessage() \\ 
				&\\
		& Ruft nächste Nachricht ab \\ \hline		
		void & sendBroadcastMessage(String message) \\ 
				&\\
		& Sendet eine Nachricht an alle Roboter \\ \hline
		String & getRobotID() \\ 
				&\\
		& Gibt den Namen des Roboters zurück. \\ \hline
		void & sendLogMessage(String logmessage) \\ 
				&\\
		& Sendet eine Nachricht an den Message Server \\ \hline
		void & sendMessage(String destination, String message) \\ 
				&\\
		& Sendet eine Nachricht an den $destination$-Roboter \\ \hline
		\end{tabular}
		\end{table}
		
		Um eine Nachricht zu empfangen, muss zuerst mit $hasMessage()$ überprüft werden ob eine Nachricht vorhanden ist. Liefert $hasMessage()$ true zurück, kann mit $getNextMessage()$ eine Nachricht abgerufen werden. Das Beispiel in Listing \ref{lst:msg} zeigt wie das geht.
		
		\begin{lstlisting}[captionpos=b, caption=Beispiel zum Abrufen einer Nachricht, label=lst:msg]
		if (hasMessage()){
          String msg = getNextMessage().getContent();
        }
		\end{lstlisting}
		
		$broadcastMessage(...)$ schickt eine Nachricht an alle mit dem selben Message-Server verbundenen Roboter.
		
		\subsection{MindroidMessage}

			Um die von $getNextMessage()$ zurückgegebene Nachricht verarbeiten zu können, muss ein zusätzlicher import hinzugefügt werden.
						\begin{center}import org.mindroid.common.messages.server.MindroidMessage;\end{center}
		
			\begin{table}[htbp]
				\begin{tabular}{|p{0.2\textwidth} p{0.75\textwidth}|}
					\hline
					\textbf{Typ} & \textbf{Methode und Beschreibung} \\ \hline
					String & getContent() \\ 
							&\\
					& Liefert den Inhalt der Nachricht zurück\\ \hline
					RobotID & getDestination() \\ 
					RobotID & getSource() \\ 
							&\\
					& Liefert die Quelle/das Ziel der Nachricht an\\ \hline		
				\end{tabular}
			\end{table}
		
	%\newpage
	\subsection{Brick}
	\subsubsection{Display}
		\begin{table}[htbp]
		\begin{tabular}{|p{0.2\textwidth} p{0.75\textwidth}|}
		\hline
		\textbf{Typ} & \textbf{Methode und Beschreibung} \\ \hline
		void & clearDisplay() \\ 
		&\\
		& Löscht den Aktuellen Inhalt des Displays \\ \hline
		void & drawString(String text)\\			
		void & drawString(String text, int row)\\
		&\\
		& Schreibt den im $text$-Parameter gegebenen Text auf das Display.\\
		& Der Parameter $row$ bestimmt die zu beschreibende Zeile. \\
		& Wird der Parameter $row$ weggelassen, wir in die Mittlere Zeile geschrieben. \\
		\hline

		\end{tabular}
		\end{table}
		\gcenter{img/ev3_display}{Koordinaten der Pixel des Displays des EV3\protect\footnotemark\ }{display}{.5\textwidth}
		
\footnotetext{ \url{https://services.informatik.hs-mannheim.de/~ihme/lectures/LEGO\_Files/01\_Anfaenger\_Graphisch\_EV3\_BadenBaden.pdf}		}

	\newpage
	\subsubsection{Buttons}
							\begin{center}import org.mindroid.impl.brick.Button;\end{center}
		\begin{table}[htbp]
		\begin{tabular}{|p{0.2\textwidth} p{0.75\textwidth}|}
		\hline
		\textbf{Typ} & \textbf{Methode und Beschreibung} \\ \hline
		boolean & isDownButtonClicked() \\ 
		boolean & isEnterButtonClicked() \\ 
		boolean & isLeftButtonClicked() \\ 
		boolean & isRightButtonClicked() \\ 
		boolean & isUpButtonClicked() \\ \hline
		\end{tabular}
		\end{table}

		Die Funktionen liefern $true$ wenn der entsprechende Button gedrückt wurde. Die Benennung der Buttons kannst du Abbildung \ref{fig:buttons} auf Seite \pageref{fig:buttons} entnehmen

		%\newpage
	\subsubsection{Sound}
		\begin{table}[htbp]
		\begin{tabular}{|p{0.2\textwidth} p{0.75\textwidth}|}
		\hline
		\textbf{Typ} & \textbf{Methode und Beschreibung} \\ \hline
		void & setSoundVolume(int volume) \\ 
		void & playBeepSequenceDown() \\ 
		void & playBeepSequenceUp() \\ 
		void & playBuzzSound() \\ 
		void & playDoubleBeep() \\ 
		void & playSingleBeep() \\ \hline
		\end{tabular}
		\end{table}
		
		Der Parameter $volume$ nimmt Werte von 0 bis 10 entgegen.
		
	\newpage
	\subsubsection{LED}		
		\begin{table}[htbp]
		\begin{tabular}{|p{0.2\textwidth} p{0.75\textwidth}|}
		\hline
		\textbf{Typ} & \textbf{Methode und Beschreibung} \\ \hline
		void & setLED(int mode) \\ 
		&\\
		& Lässt die LED des EV3 im angegebenen Modus leuchten\\
		& Der Parameter $mode$ kann entweder als Ganzzahl von 0 bis 9 oder als Konstante angegeben werden. \\
		& Siehe Tabelle \ref{tab:led}\\
		\hline
		
		\end{tabular}
		\end{table}
		
		
		
		\begin{table}[htbp]
		\caption{Funktion der einzelnen Modi der LED}
		\begin{center}
		\begin{tabular}{r|l|l|l}
		
		\multicolumn{2}{c|}{\textbf{Modus (Parameter $mode$)}}   & \textbf{Farbe} & \textbf{Intervall} \\ 
		\multicolumn{1}{l|}{Wert} & Konstante &  &  \\ \hline
		0 & LED\_OFF & Aus & Aus \\ 
		1 & LED\_GREEN\_ON & Grün & Dauer \\ 
		2 & LED\_GREEN\_BLINKING & Grün & Blinken \\ 
		3 & LED\_GREEN\_FAST\_BLINKING & Grün & Schnell Blinken \\ 
		4 & LED\_YELLOW\_ON & Gelb & Dauer \\ 
		5 & LED\_YELLOW\_BLINKING & Gelb & Blinken \\ 
		6 & LED\_YELLOW\_FAST\_BLINKING & Gelb & Schnell Blinken \\ 
		7 & LED\_RED\_ON & Rot & Dauer \\ 
		8 & LED\_RED\_BLINKING & Rot & Blinken \\ 
		9 & LED\_RED\_FAST\_BLINKING & Rot & Schnell Blinken \\ 
		\end{tabular}
		\end{center}
		\label{tab:led}
		\end{table}
		
		\newpage
		\section{EV3 Tasten}
		Abbildung \ref{fig:buttons} zeigt dir wie die Tasten am EV3-Brick genannt werden. Die Enter-Taste wird zum Bestätigen genutzt, mit der Escape-Taste, geht es ein Menü zurück. 
		\gcenter{img/ev3_buttons.png}{EV3-Tastenbelegung\protect\footnotemark\ }{buttons}{.5\textwidth}
		\footnotetext{ Quelle http://www.ev3dev.org/images/ev3/labeled-buttons.png}
		
		Die Bedeutung der Tasten kannst du der folgenden Aufzählung entnehmen. 
		\begin{enumerate}
			\item \textbf{Escape / Zurück}
			\item \textbf{Up / Hoch}
			\item \textbf{Left / Links}
			\item \textbf{Enter / Bestätigen}
			\item \textbf{Right / Rechts}  
			\item \textbf{Down / Unten}  
		\end{enumerate}
	
	\ifthenelse{\boolean{withJava}}{
		\section{Kurze Übersicht über Java}
		\arrayrulecolor{white}
		\begin{table}[h]
			\renewcommand{\arraystretch}{1.5}
			\begin{center}
				\begin{tabular}{|p{0.3\textwidth}|p{0.7\textwidth}|}
					\hline
					Schleife mit Bedingung& while(Bedingung) \{Programmcode\} \tabularnewline
					\space & \scriptsize Beispiel: while(i<100)\{...\} \tabularnewline
					\hline
					Zählschleife& for(Start; Bedingung; Zählschritte) \{Programmcode\} \tabularnewline
					\space & \scriptsize Beispiel: for(int i=0;i<10;i++)\{...\} \tabularnewline
					\hline
					Bedingung& if(Bedingung) \tabularnewline
					\space& \{wenn die Bedingung wahr ist, wird dieser Code ausgeführt\} \tabularnewline
					\space & else \tabularnewline
					\space & \{wenn die Bedingung falsch ist, wird dieser Code ausgeführt\}
					\tabularnewline
					\hline			
				\end{tabular}
			\end{center}
		\end{table}
	}{}
		
	}{}
	\ifthenelse{\boolean{devDoc}}{
			\section{PAN Einrichtung}
		\label{sec:pan}
		Wird im Hauptmenü noch nicht die richtige IP-Adresse angezeigt, müssen zuerst die PAN\footnote{PAN = Personal Area Network}-Einstellungen korrigiert werden.
		
		\begin{enumerate}	
			\begin{minipage}{.45\textwidth}
				\item Dazu musst du zuerst in das \textbf{PAN-Menü} des Roboters navigieren. Wechsle mit den Links-/Rechts-Tasten bis du den Menüpunkt \textbf{PAN} siehst und betätige die \textbf{Auswahltaste}.
			\end{minipage}
			\hfill
			\begin{minipage}{.45\textwidth}
				\includegraphics[width=.8\textwidth]{img/ev3_pan.png}
			\end{minipage}
			
			\item Nun navigierst du durch das Menü des Roboters wie auf den Bildern zu sehen und bestätigst jeweils mit der Auswahltaste: \textbf{USB-Client - Address - Advanced}
			
			\includegraphics[width=.3\textwidth]{img/ev3_pan_usb.png}
			\includegraphics[width=.3\textwidth]{img/ev3_pan_usb_address.png}		
			\includegraphics[width=.3\textwidth]{img/ev3_pan_usb_advanced.png}
			
			Nun musst du die IP Adresse 192.168.42.253 einstellen. Dazu navigierst du mit den rechts-/links-Tasten zu den einzelnen Ziffern und änderst deren Wert mit den oben-/unten-Tasten. Orientiere dich an den Bildern! Am Ende bestätigst du wieder mit der Enter-Taste. 
			
			\includegraphics[width=.3\textwidth]{img/ev3_pan_usb_setip1.png}
			\includegraphics[width=.3\textwidth]{img/ev3_pan_usb_setip2.png}
			\includegraphics[width=.3\textwidth]{img/ev3_pan_usb_isset.png}
			
			\begin{minipage}{.6\textwidth}
				\item Mit der \textbf{Zurück}-Taste kommst du wieder in das Hauptmenü und die Einstellungen werden übernommen.
			\end{minipage}
			\hfill
			\begin{minipage}{.3\textwidth}
				\includegraphics[width=\textwidth]{img/ev3_pan_usb_restart.png}
			\end{minipage}
			%		\\Nun kannst du wieder zu Punkt 3 auf Seite \pageref{sec:afterpan} wechseln.
		\end{enumerate}
		
		\section{Troubleshooting}
		\subsection{Installation über WLAN funktioniert nicht}
		Im Message-Server über \bfcode{File->Connected Devices }schauen ob alle Smartphones in der Liste auftauchen und der ADB-state auf \bfcode{connected} steht. 
		Ist dies nicht der Fall, tippe in der App auf \bfcode{TRENNEN} und stelle die Verbindung danach erneut her.
		Falls das nicht klappt, kontaktiere einen Betreuer. 
		Falls das Installieren per WLAN gar nicht mehr funktioniert, kann jederzeit eine USB-Verbindung zwischen Smartphone und PC hergestellt werden und darüber die App installiert werden.
		
		
		
		\section{Sensorbelegung}
		
		Tabelle	\ref{tab:sensors} zeigt die standardmäßige Sensorbelegung, wie sie in der App unter ``Mein Roboter'' definiert sein muss.\\
			
	
		\begin{table}[h]
			\begin{center}
				\begin{tabular}{r|l|l}						
					\textbf{Anschluss} & \textbf{Sensortyp} & \textbf{Modus} \\ \hline
					1 & Farbe & ColorID \\ 
					2 & Ultraschall & Distance \\ 
					3 & Gyroskop & Angle \\ 
					4 & Farbe & ColorID \\ 
				\end{tabular}
				\caption{Sensorbelegung}
				\label{tab:sensors}
			\end{center}
		\end{table}
	
		Tabelle \ref{tab:motors} zeigt den standardmäßigen Motoranschluss, wie er in der App unter ``Mein Roboter'' definiert sein muss.\\
		
		\begin{table}[h]
			\begin{center}
				\begin{tabular}{r|l}						
					\textbf{Anschluss} & \textbf{Motor} \\ \hline
					A & Large Regulated Motor \\ 
					B & - \\ 
					C & - \\ 
					D & Large Regulated Motor \\ 
				\end{tabular}
			\end{center}
			\caption{Motorbelegung}
			\label{tab:motors}
			\end{table}
	\end{document}		
	}{}
	
	\ifthenelse{\boolean{intro}}{
		\input{tasks/intro_helloWorld}
		\input{tasks/intro_parkSensor}
		\input{tasks/intro_colorSensor}
		\input{tasks/intro_helloWorldCoord}
	}{}
	
	\ifthenelse{\boolean{tasks}}{	
		\input{tasks/mower}		
		\input{tasks/wallPingPong}
		\input{tasks/platooning}
		\input{tasks/wallPingPongCoord}
		\input{tasks/dancing}
	}{}





\cfoot{\textcolor{lightgray} \today}




\end{document}
