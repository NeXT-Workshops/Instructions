	
	
	\section{Einführung}
	In dieser Übersicht werden die Funktionen, die zur Steuerung der Roboter zur Verfügung stehen erklärt.
	Zur Verdeutlichung ein kleines Beispiel:
	
		\begin{table}[htbp]
		\begin{tabular}{|p{0.2\textwidth} p{0.75\textwidth}|}
			\hline
			\textbf{Typ} & \textbf{Methode und Beschreibung} \\ \hline
			void & delay(long milliseconds) \\ 
			&\\
			& Verzögert die Ausführung um die angegebene Zeitspanne (Milisekunden)\\ \hline
		\end{tabular}
		\end{table}
	
	Die Spalte \textit{Typ }gibt an, welchen Typ der Rückgabewert der Funktion hat. \textit{void} bedeutet, dass kein Wert zurück gegeben wird. 
	In der Klammer hinter dem Funktionsnamen wird angegeben, welche Parameter die Funktion erwartet, und von welchem Typ diese sein müssen. In unserem Beispiel bedeutet dies, dass die \textit{delay}-Methode einen Parameter vom Typ \textit{long} (ganzzahliger Wert) erwartet, welcher \textit{milliseconds }genannt wird. 
	Ein möglicher Funktionsaufruf sieht wie folgt aus:
	
	\begin{lstlisting}
	public void run(){	
		delay(1000);
	}
	\end{lstlisting}
	
	Dabei wird die delay-Methode mit 1000 als Parameter aufgerufen. Das bedeutet, die Ausführung wird um $1000ms$ ($= 1s$) verzögert.
	
	
	\subsection{isInterrputed}
	Damit die Ausführung des Programms auch in Schleifen unterbrochen werden kann, sollte jede Schleife die isInterrupted-Methode abfragen. 
	
	Beispiel:
	\begin{lstlisting}
		public void run(){	
			while(!isInterrupted()){
				// Schleifeninhalt
			}
			for(int i=0; i<10 && !isInterrupted(); i++){
				// Schleifeninhalt
			}
		}
	\end{lstlisting}
	
	

	\section{Wichtige Funktionen}
	Hier eine kleine Übersicht über die wichtigsten Funktionen beim Programmieren der Roboter.
	
	\subsection{Fahren}
		\begin{center}\texttt{import org.mindroid.api.ImperativeWorkshopAPI}\end{center}
		
		Mögliche Eingabewerte für den $speed$-Parameter liegen zwischen 0 und 1000.
		Eine maximale Geschwindigkeit von $300$ sollte ausreichen. Niedrigere Geschwindigkeiten schonen den Akku.	
		Die Distanz wird im $distance$-Parameter immer als Kommazahl in Zentimetern (cm) angegeben (z.B.: 20cm werden als $20.0f$ angegeben)
	
		\begin{table}[htbp]
		\begin{tabular}{|p{0.2\textwidth} p{0.75\textwidth}|}
		\hline
		\textbf{Typ} & \textbf{Methode und Beschreibung} \\ \hline
		void & setMotorSpeed(int speed) \\
		& \\
		& Bestimmt die Geschwindigkeit für Fahrmethoden ohne $speed$-Parameter. \\ \hline
		void & forward() \\ 
		void & backward() \\ & \\
		& Fahren mit der von $setMotorSpeed(...)$ gesetzten Geschwindigkeit. \\ \hline
		void & driveDistanceForward(float distance) \\
		void & driveDistanceBackward(float distance) \\ & \\
		& Fahren mit der von $setMotorSpeed(...)$ gesetzten Geschwindigkeit\\ 
		& Die Distanz muss in Zentimetern angegeben werden. \\ \hline
		
		void & forward(int speed) \\ 
		void & backward(int speed)  \\ 
		void & driveDistanceForward(float distance, int speed) \\ 
		void & driveDistanceBackward(float distance, int speed) \\ 		
		& \\
		& Wie oben, nur dass der $speed$-Parameter die von $setMotorSpeed()$ gesetzte Geschwindigkeit überschreibt. Nach Beendigung des Aufrufs, wird wieder die vorher gesetzte Geschwindigkeit genutzt.\\ \hline

		void & turnLeft(int degrees) \\ 
		void & turnRight(int degrees) \\ 
		void & turnLeft(int degrees, int speed) \\ 
		void & turnRight(int degrees, int speed) \\ 
		& \\
		& Dreht den Roboter um den im $degrees$-Parameter bestimmten Wert. \\
		& Der $Speed$-Parameter verhält sich wie bei den anderen Methoden. \\ \hline
		
		void & stop() \\ 
		& \\
		& Stoppt sofort alle Motoren.	\\ \hline
		\end{tabular}
		\end{table}
		
		\newpage
	\subsection{Sensoren}
			\begin{center}\texttt{import org.mindroid.api.ImperativeWorkshopAPI}\end{center}
		\begin{table}[htbp]
		\begin{tabular}{|p{0.2\textwidth} p{0.75\textwidth}|}
		\hline
		\textbf{Typ} & \textbf{Methode und Beschreibung} \\ \hline
		float & getAngle() \\ 
		&\\
		& Liefert den Winkel des Gyrosensors in Grad\\ \hline
		float & getDistance() \\ 
		&\\
		& Liefert die vom Ultraschallsensor gemessene Distanz in Zentimetern\\ \hline		
		Colors & getLeftColor() \\ 
		Colors & getRightColor() \\
		&\\
		& Liefert den Wert des Linken/Rechten Farbsensors\\ 
		& Farbwerte: Colors.BLACK, Colors.BLUE, Colors.BROWN, Colors.GREEN, \\ &Colors.RED, Colors.WHITE, Colors.YELLOW, Colors.NONE\\ \hline
		\end{tabular}
		\end{table}

	\subsection{Kommunikation}
				\begin{center}\texttt{import org.mindroid.api.ImperativeWorkshopAPI}\end{center}
		\begin{table}[htbp]
		\begin{tabular}{|p{0.2\textwidth} p{0.75\textwidth}|}
		\hline
		\textbf{Typ} & \textbf{Methode und Beschreibung} \\ \hline
		boolean & hasMessage() \\ 
				&\\
		& Prüft ob Nachricht vorhanden ist \\ \hline
		MindroidMessage & getNextMessage() \\ 
				&\\
		& Ruft nächste Nachricht ab \\ \hline		
		void & sendBroadcastMessage(String message) \\ 
				&\\
		& Sendet eine Nachricht an alle Roboter \\ \hline
		String & getRobotID() \\ 
				&\\
		& Gibt den Namen des Roboters zurück. \\ \hline
		void & sendLogMessage(String logmessage) \\ 
				&\\
		& Sendet eine Nachricht an den Message Server \\ \hline
		void & sendMessage(String destination, String message) \\ 
				&\\
		& Sendet eine Nachricht an den $destination$-Roboter \\ \hline
		\end{tabular}
		\end{table}
		
		Um eine Nachricht zu empfangen, muss zuerst mit $hasMessage()$ überprüft werden ob eine Nachricht vorhanden ist. Liefert $hasMessage()$ true zurück, kann mit $getNextMessage()$ eine Nachricht abgerufen werden. Das Beispiel in Listing \ref{lst:msg} zeigt wie das geht.
		
		\begin{lstlisting}[captionpos=b, caption=Beispiel zum Abrufen einer Nachricht, label=lst:msg]
		if (hasMessage()){
          String msg = getNextMessage().getContent();
        }
		\end{lstlisting}
		
		$broadcastMessage(...)$ schickt eine Nachricht an alle mit dem selben Message-Server verbundenen Roboter.
		
		\subsection{MindroidMessage}

			Um die von $getNextMessage()$ zurückgegebene Nachricht verarbeiten zu können, muss ein zusätzlicher import hinzugefügt werden.
						\begin{center}import org.mindroid.common.messages.server.MindroidMessage;\end{center}
		
			\begin{table}[htbp]
				\begin{tabular}{|p{0.2\textwidth} p{0.75\textwidth}|}
					\hline
					\textbf{Typ} & \textbf{Methode und Beschreibung} \\ \hline
					String & getContent() \\ 
							&\\
					& Liefert den Inhalt der Nachricht zurück\\ \hline
					RobotID & getDestination() \\ 
					RobotID & getSource() \\ 
							&\\
					& Liefert die Quelle/das Ziel der Nachricht an\\ \hline		
				\end{tabular}
			\end{table}
		
	%\newpage
	\subsection{Brick}
	\subsubsection{Display}
		\begin{table}[htbp]
		\begin{tabular}{|p{0.2\textwidth} p{0.75\textwidth}|}
		\hline
		\textbf{Typ} & \textbf{Methode und Beschreibung} \\ \hline
		void & clearDisplay() \\ 
		&\\
		& Löscht den Aktuellen Inhalt des Displays \\ \hline
		void & drawString(String text)\\			
		void & drawString(String text, int row)\\
		&\\
		& Schreibt den im $text$-Parameter gegebenen Text auf das Display.\\
		& Der Parameter $row$ bestimmt die zu beschreibende Zeile. \\
		& Wird der Parameter $row$ weggelassen, wir in die Mittlere Zeile geschrieben. \\
		\hline

		\end{tabular}
		\end{table}
		\gcenter{img/ev3_display}{Koordinaten der Pixel des Displays des EV3\protect\footnotemark\ }{display}{.5\textwidth}
		
\footnotetext{ \url{https://services.informatik.hs-mannheim.de/~ihme/lectures/LEGO\_Files/01\_Anfaenger\_Graphisch\_EV3\_BadenBaden.pdf}		}

	\newpage
	\subsubsection{Buttons}
							\begin{center}import org.mindroid.impl.brick.Button;\end{center}
		\begin{table}[htbp]
		\begin{tabular}{|p{0.2\textwidth} p{0.75\textwidth}|}
		\hline
		\textbf{Typ} & \textbf{Methode und Beschreibung} \\ \hline
		boolean & isDownButtonClicked() \\ 
		boolean & isEnterButtonClicked() \\ 
		boolean & isLeftButtonClicked() \\ 
		boolean & isRightButtonClicked() \\ 
		boolean & isUpButtonClicked() \\ \hline
		\end{tabular}
		\end{table}

		Die Funktionen liefern $true$ wenn der entsprechende Button gedrückt wurde. Die Benennung der Buttons kannst du Abbildung \ref{fig:buttons} auf Seite \pageref{fig:buttons} entnehmen

		%\newpage
	\subsubsection{Sound}
		\begin{table}[htbp]
		\begin{tabular}{|p{0.2\textwidth} p{0.75\textwidth}|}
		\hline
		\textbf{Typ} & \textbf{Methode und Beschreibung} \\ \hline
		void & setSoundVolume(int volume) \\ 
		void & playBeepSequenceDown() \\ 
		void & playBeepSequenceUp() \\ 
		void & playBuzzSound() \\ 
		void & playDoubleBeep() \\ 
		void & playSingleBeep() \\ \hline
		\end{tabular}
		\end{table}
		
		Der Parameter $volume$ nimmt Werte von 0 bis 10 entgegen.
		
	\newpage
	\subsubsection{LED}		
		\begin{table}[htbp]
		\begin{tabular}{|p{0.2\textwidth} p{0.75\textwidth}|}
		\hline
		\textbf{Typ} & \textbf{Methode und Beschreibung} \\ \hline
		void & setLED(int mode) \\ 
		&\\
		& Lässt die LED des EV3 im angegebenen Modus leuchten\\
		& Der Parameter $mode$ kann entweder als Ganzzahl von 0 bis 9 oder als Konstante angegeben werden. \\
		& Siehe Tabelle \ref{tab:led}\\
		\hline
		
		\end{tabular}
		\end{table}
		
		
		
		\begin{table}[htbp]
		\caption{Funktion der einzelnen Modi der LED}
		\begin{center}
		\begin{tabular}{r|l|l|l}
		
		\multicolumn{2}{c|}{\textbf{Modus (Parameter $mode$)}}   & \textbf{Farbe} & \textbf{Intervall} \\ 
		\multicolumn{1}{l|}{Wert} & Konstante &  &  \\ \hline
		0 & LED\_OFF & Aus & Aus \\ 
		1 & LED\_GREEN\_ON & Grün & Dauer \\ 
		2 & LED\_GREEN\_BLINKING & Grün & Blinken \\ 
		3 & LED\_GREEN\_FAST\_BLINKING & Grün & Schnell Blinken \\ 
		4 & LED\_YELLOW\_ON & Gelb & Dauer \\ 
		5 & LED\_YELLOW\_BLINKING & Gelb & Blinken \\ 
		6 & LED\_YELLOW\_FAST\_BLINKING & Gelb & Schnell Blinken \\ 
		7 & LED\_RED\_ON & Rot & Dauer \\ 
		8 & LED\_RED\_BLINKING & Rot & Blinken \\ 
		9 & LED\_RED\_FAST\_BLINKING & Rot & Schnell Blinken \\ 
		\end{tabular}
		\end{center}
		\label{tab:led}
		\end{table}
		
		\newpage
		\section{EV3 Tasten}
		Abbildung \ref{fig:buttons} zeigt dir wie die Tasten am EV3-Brick genannt werden. Die Enter-Taste wird zum Bestätigen genutzt, mit der Escape-Taste, geht es ein Menü zurück. 
		\gcenter{img/ev3_buttons.png}{EV3-Tastenbelegung\protect\footnotemark\ }{buttons}{.5\textwidth}
		\footnotetext{ Quelle http://www.ev3dev.org/images/ev3/labeled-buttons.png}
		
		Die Bedeutung der Tasten kannst du der folgenden Aufzählung entnehmen. 
		\begin{enumerate}
			\item \textbf{Escape / Zurück}
			\item \textbf{Up / Hoch}
			\item \textbf{Left / Links}
			\item \textbf{Enter / Bestätigen}
			\item \textbf{Right / Rechts}  
			\item \textbf{Down / Unten}  
		\end{enumerate}
	
	\ifthenelse{\boolean{withJava}}{
		\section{Kurze Übersicht über Java}
		\arrayrulecolor{white}
		\begin{table}[h]
			\renewcommand{\arraystretch}{1.5}
			\begin{center}
				\begin{tabular}{|p{0.3\textwidth}|p{0.7\textwidth}|}
					\hline
					Schleife mit Bedingung& while(Bedingung) \{Programmcode\} \tabularnewline
					\space & \scriptsize Beispiel: while(i<100)\{...\} \tabularnewline
					\hline
					Zählschleife& for(Start; Bedingung; Zählschritte) \{Programmcode\} \tabularnewline
					\space & \scriptsize Beispiel: for(int i=0;i<10;i++)\{...\} \tabularnewline
					\hline
					Bedingung& if(Bedingung) \tabularnewline
					\space& \{wenn die Bedingung wahr ist, wird dieser Code ausgeführt\} \tabularnewline
					\space & else \tabularnewline
					\space & \{wenn die Bedingung falsch ist, wird dieser Code ausgeführt\}
					\tabularnewline
					\hline			
				\end{tabular}
			\end{center}
		\end{table}
	}{}
