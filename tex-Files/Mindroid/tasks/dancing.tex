\section{Dancing Robots}
	\ifthenelse{\boolean{tasks}}{
		%\easygcenter{.8\textwidth}{img/task_chachacha.png}
		Beim Cha-Cha-Cha gibt es die Tanzfigur “Verfolgung”. Dabei verfolgt jeweils ein Tanzpartner den anderen, bis beide sich umdrehen und die Rollen wechseln. Diese Figur ist tatsächlich nicht sehr weit vom Platooning-Beispiel aus der vorherigen Aufgabe entfernt.
		Der Ablauf soll dieses Mal wie folgt aussehen:
		\begin{enumerate}
		\item  Roboter A übernimmt zunächst das Kommando und fährt voraus, während Roboter B einen möglichst gleichbleibenden Abstand hält.
		\item  Roboter A beschließt nach einer gewissen Zeit, dass nun die Drehung folgt. Er stoppt und sendet eine “Drehen”-Nachricht an Roboter B.
		\item  Roboter B stoppt, wendet und sendet Roboter A eine “Gedreht”-Nachricht.
		\item  Daraufhin dreht Roboter A ebenfalls um 180°.
		\item  Nun tauschen Roboter A und B die Rollen: Roboter B fährt voraus und gibt den Ton bis zur nächsten Drehung an.
		\end{enumerate}
	}{}
	\ifthenelse{\boolean{scoring}}{
		Bewertungskriterien
		\begin{itemize}
		\item (1P) Einmal Verfolgung hin (Roboter A ist der Führende) und einmal Verfolgung zurück (Roboter B ist der Führende) und dabei keine Kollision!
		\item (1P) Es sollte klar erkennbar sein (bspw. per LED-Statusleuchte), wer aktuell der “Führende” ist.
		\end{itemize}
	}{}
	\ifthenelse{\boolean{solution}}{
		\sol{Follow}
		\sol{FollowA}
		\sol{FollowB}
	}{}