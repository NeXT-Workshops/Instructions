\section{Platooning}
	\ifthenelse{\boolean{tasks}}{
	%	\easygcenter{.8\textwidth}{img/task_platooning.jpg}
		In dieser Aufgabe geht es darum, zwei Roboter hintereinander her fahren zu lassen, ohne dass es einen Auffahrunfall gibt. Aktuell forschen zahlreiche Unis und Unternehmen unter dem Schlagwort Platooning an genau dieser Problemstellung bei echten LKWs und PKWs: Die Fahrzeuge fahren dabei so nahe, dass sie den Windschatten des Vorausfahrenden ausnutzen können.
		
		Roboter A und B werden hintereinander platziert, sodass sie in die gleiche Richtung blicken. Ziel ist es zunächst, dass Roboter B den Abstand zu Roboter A in einem bestimmen Toleranzbereich hält. Die Distanzangaben im Folgenden sind nur mögliche Werte - du bestimmst selbst, was geeignete Grenzwerte sind.
		\begin{enumerate}
		\item Roboter A fährt los. Sobald der Abstand zwischen Roboter A und Roboter B größer als 35cm wird, beginnt Roboter B aufzuschließen.
		\item Wird der Abstand kleiner als 25cm, hält Roboter B die Geschwindigkeit von Roboter A .
		\item Wird der Abstand kleiner als 15cm, lässt Roboter B sich zurückfallen (oder fährt sogar rückwärts).
		\end{enumerate}
	}{}
	\ifthenelse{\boolean{scoring}}{
		Bewertungskriterien
		\begin{itemize}
		\item (1P) Roboter A und B fahren mind. 1 Meter hintereinander her ohne Kollisionen (weder mit anderen Robotern noch mit der Wand).
		\item (1P) Abstand bleibt in einem von euch zuvor bestimmten Bereich (Schnur), wenn beide Roboter vorwärts fahren.
		\item (1P) Vergleich: Geringstmöglicher Abstand, bei dem keine Kollision stattfindet.
		\end{itemize}
	}{}
	\ifthenelse{\boolean{solution}}{
		\sol{Platooning}
		\sol{PlatooningLeader}
		\sol{PlatooningFollower}
	}{}