\subsection{Die Farbsensoren - Farbe messen}
	In dieser Aufgabe lernst du die Farbsensoren des Roboters kennen. Der folgende Quelltext liest kontinuierlich den aktuell gemessenen Farbwert des linken und rechten Lichtsensors aus (\bfcode{getLeftColor()} bzw. \bfcode{getRightColor()} in Zeilen 16 und 17).
	\\
	
	\lstinputlisting[firstline=3]{\solpath/ColorTest.java}
	
	\begin{itemize}
	\item Die Methode \bfcode{describeColor} (Zeilen 29-39) zeigt, wie du den Rückgabewert in einen lesbaren Text umwandelst.
	\item In den Zeilen 20-21 siehst du, wie man auf dem Display mehrzeiligen Text ausgeben kann. Die Buchstaben haben jeweils eine Höhe von 16 Pixeln, sodass die zweite Zeile an der y-Position 17 und die dritte Zeile an der y-Position 33 beginnt.
	\item Um die Qualität der Farbmessung näher zu betrachten, haben wir für dich Farbtafeln mit allen sieben unterstützten Farben des EV3-Lichtsensors vorbereitet. Bei welchen Farben funktioniert die Erkennung gut, bei welchen eher weniger?
	\item Der Farbsensor kann auch zur Erkennung von Abgründen eingesetzt werden: Welche Farbwerte werden gemessen, wenn der Roboter auf der Tischplatte steht und wenn die Farbsensoren über den Tischrand ragen?
	\end{itemize}