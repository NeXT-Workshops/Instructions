	\section{Hello World}
	In der Informatik ist es üblich, ein Hallo-Welt-Programm\footnote{https://de.wikipedia.org/wiki/Hallo-Welt-Programm} zu schreiben, wenn man eine Programmierumgebung kennenlernt. Deshalb fangen wir damit an. 
	Nachdem du den Roboter erfolgreich eingerichtet und das erste Mal getestet hast, gehen wir jetzt daran, uns den Quelltext näher anzusehen.
	\lstinputlisting[firstline=3]{\solpath/HelloWorld.java}
	Das Verhalten des Roboters befindet sich in der \textbf{run-Methode }(Zeilen 13-16). Wenn ein Mindroid-Programm ausgeführt wird, werden die Befehle in dieser Methode nacheinander abgearbeitet.
		\begin{itemize}
		\item{\bfcode{clearDisplay()}} löscht den Display-Inhalt
		\item{\bfcode{drawString(text, textsize, xPosition, yPosition)}} schreibt einen gegebenen Text (\bfcode{text}) an die gegebenen Koordinaten (\bfcode{xPosition, yPosition}) und verwendet dabei die definierte Schriftgröße (\bfcode{textsize}).
		\item Der Konstruktor in den Zeilen 8 bis 10 gibt unserem Programm einen Namen (Zeile 8). Mit dem Aufruf von \bfcode{super("Hello World")} bestimmen wir, unter welcher Bezeichnung unser Programm später ausgewählt werden kann. Es ist sinnvoll an beiden Stellen aussagekräftige Bezeichnungen zu verwenden.
		\end{itemize}
		Daneben gibt es noch die sogenannten “\textbf{imports}”. Da die Programm-Bibliotheken der MindroidApp sehr groß sind, hat jede Klasse einen ausführlichen Namen, der dabei hilft, den Überblick zu bewahren. Der Teil vor dem Klassennamen heißt \textbf{Paket} (engl. package). Zum Beispiel ist die Klasse \textbf{ImperativeWorkshopAPI} im Paket \textbf{org.mindroid.api.}