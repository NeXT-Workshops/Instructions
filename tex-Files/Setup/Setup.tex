%% This is file `DEMO-TUDaPub.tex' version 2.09 (2020/03/13),
%% it is part of
%% TUDa-CI -- Corporate Design for TU Darmstadt
%% ----------------------------------------------------------------------------
%%
%%  Copyright (C) 2018--2020 by Marei Peischl <marei@peitex.de>
%%
%% ============================================================================
%% This work may be distributed and/or modified under the
%% conditions of the LaTeX Project Public License, either version 1.3c
%% of this license or (at your option) any later version.
%% The latest version of this license is in
%% http://www.latex-project.org/lppl.txt
%% and version 1.3c or later is part of all distributions of LaTeX
%% version 2008/05/04 or later.
%%
%% This work has the LPPL maintenance status `maintained'.
%%
%% The Current Maintainers of this work are
%%   Marei Peischl <tuda-ci@peitex.de>
%%   Markus Lazanowski <latex@ce.tu-darmstadt.de>
%%
%% The development respository can be found at
%% https://github.com/tudace/tuda_latex_templates
%% Please use the issue tracker for feedback!
%%
%% ============================================================================
%%
% !TeX program = lualatex
%%

\documentclass[
	ngerman,
	accentcolor=1c,% Farbe für Hervorhebungen auf Basis der Deklarationen in den Corporate Design Richtlinien
%	logofile=example-image, %Falls die Logo Dateien nicht vorliegen
	]{tudapub}

\usepackage[english, main=ngerman]{babel}
\usepackage[babel]{csquotes}

\usepackage{biblatex}
\bibliography{DEMO-TUDaBibliography}

%Formatierungen für Beispiele in diesem Dokument. Im Allgemeinen nicht notwendig!
\let\file\texttt
\let\code\texttt
\let\pck\textsf
\let\cls\textsf

\usepackage{hologo}




\begin{document}

%Zusätzliche Metadaten für PDF/A. In diesem Fall notwendig, weil Titel ein Makro enthält.
\Metadata{
	author=NeXT,
	title=Space Workshop Rating,
	%subject=Basisdokumentation und Template zur Nutzung der tudapub-Dokumentenkasse,
	%date=2019-04-29,
	%keywords=TU Darmstadt \sep Corporate Design \sep LaTeX
}




\title{Lego EV3\newline Lejos-Einrichtung}
\subtitle{NeXT Generation on Campus}
%\author{Marei Peischl\thanks{pei\TeX{} \TeX{}nical Solutions}\and der \TeX-Löwe}
\date{}
%\titleimage{
%%	%Folgende Box kann selbstverständlich durch ein mit \includegraphics geladenes Bild ersetzt werden.
%\bigskip
%\bigskip
%\bigskip
%\bigskip
%\bigskip
%\bigskip
%\bigskip
%\bigskip
%\bigskip
%\bigskip
%\bigskip
%\includegraphics[width=\textwidth]{../control/images/title.jpg}
%	%\color{black!30}\rule{\width}{\height}
%}


%Varianten der Infoboxen
\addTitleBox{\includegraphics[width=\linewidth]{ist_logo.pdf}}
%\addTitleBoxLogo{example-image}
%\addTitleBoxLogo*{\includegraphics[width=.3\linewidth]{example-image}}



\maketitle


\newpage


\tableofcontents

\section{Ben\"otigtes Material}
F\"ur die verschiedenen Workshops ist verschieden viel Material notwendig. Als Beispiele sind die von uns erfolgreich genutzten Hardware-Komponenten gelistet, mit anderen Ger\"aten sollte die Software nat\"urlich genauso laufen.
\subsection{Grundlagen}
\begin{description}
	\item[PC]
	\item[EV3]
	\item[microSD] m\"oglichst maximal 8GB
\end{description}

\subsection{F\"ur den Mindroid-Workshop}


\begin{description}
	\item[WLAN-Dongle] z.B. Edimax EW-7811Un
	\item[Smartphone] z.B. Google Nexus 5
	\item[Router] z.B. 
\end{description}

\subsection{Ben\"otigte Git-Repositories}
Um den Space-Workshop durchzuf\"uhren sind keine weiteren Dateien von uns notwendig, sie dienen aber als Unterst\"utzung. F\"ur den Mindroid-Workshop ist mindestens das dritte repository zwingend notwendig. Falls die Latex-Dateien ver\"andert werden sollen, ist es wichtig, dass sich die repositories im gleichen \"Uberordner befinden.\newline
In diesem
\href{https://download.oracle.com/otn/java/jdk/8u231-b11/5b13a193868b4bf28bcb45c792fce896/jdk-8u231-windows-x64.exe}{\textbf{repository}\footnote{$https://download.oracle.com/otn/java/jdk/8u231-b11/5b13a193868b4bf28bcb45c792fce896/jdk-8u231-windows-x64.exe$}}
befinden sich für den Space-Workshop eine Dokumentation mit den wichtigsten Befehlen sowie eine m\"ogliche Punktewertung des Szenarios, sowie eine Dokumentation und m\"ogliche Aufgaben und L\"osungen f\"ur den Mindroid-Workshop.
\href{https://download.oracle.com/otn/java/jdk/8u231-b11/5b13a193868b4bf28bcb45c792fce896/jdk-8u231-windows-x64.exe}{\textbf{Hier}\footnote{$https://download.oracle.com/otn/java/jdk/8u231-b11/5b13a193868b4bf28bcb45c792fce896/jdk-8u231-windows-x64.exe$}}
befinden sich APIs zur Vereinfachung der Bearbeitung, das bedeutet, dass mit verschiedenen Niveaustufen, unterschiedlich viele Hilfestellungen gegeben werden. Diese k\"onnen einfach heruntergeladen werden und entsprechend in intelliJ ge\"offnet werden.


\section{Einrichtung der PCs- Grundlagen}
Im Folgenden werden die verschiedenen Schritte für die Einrichtung der PCs erl\"autert. Diese Anleitung gilt ausschlie\ss{}lich f\"ur Windows-PCs.

\subsection{Installation der Software}
F\"ur die Kommunikation zu den Robotern wird vorerst die Installation folgender Programme ben\"otigt. Der Link f\"uhrt direkt zum Download, sodass die Installation entsprechend der weiteren Anweisungen durchgef\"uhrt werden kann.\newline

\subsubsection{Java Developement Kit}
Um den Programmcode f\"ur den Roboter zu schreiben, ist ein JDK notwendig, welches, falls noch nicht vorhanden, 
\href{https://download.oracle.com/otn/java/jdk/8u231-b11/5b13a193868b4bf28bcb45c792fce896/jdk-8u231-windows-x64.exe}{\textbf{hier}\footnote{$https://download.oracle.com/otn/java/jdk/8u231-b11/5b13a193868b4bf28bcb45c792fce896/jdk-8u231-windows-x64.exe$}}
mit einem kostenlosen Oracle-Konto heruntergeladen werden kann. Falls sich auf dem PC ein \"alteres 32-Bit Betriebssystem befindet ist 
\href{https://download.oracle.com/otn/java/jdk/8u231-b11/5b13a193868b4bf28bcb45c792fce896/jdk-8u231-windows-i586.exe}{\textbf{dieser Link}\footnote{$https://download.oracle.com/otn/java/jdk/8u231-b11/5b13a193868b4bf28bcb45c792fce896/jdk-8u231-windows-i586.exe$}}
richtig. Die Installation kann entsprechend der weiteren Anweisungen durchgef\"uhrt werden.\newline
Damit die folgenden Programme Java finden, m\"ussen die Systemvariablen gesetzt werden. Zu Diesen gelangt man \"uber \textit{Systemsteuerung} \rightarrow{} \textit{System und Sicherheit} \rightarrow{} \textit{System} \rightarrow{} \textit{erweiterte Systemeinstellungen}. Dort gibt es das Feld \textit{Umgebungsvariablen}, mit dem man zu einem Fenster mit den \textit{Systemvariablen} gelangt. Dort muss der Path mit \textit{Bearbeiten} erg\"anzt werden. Im Feld \textit{Wert der Variablen} muss folgender Befehl hinzugef\"ugt werden:\newline
F\"ur 64-Bit-Systeme:\\$;C:\backslash ProgramFiles\backslash Java\backslash jdk1.8.0_231\backslash bin$\newline
F\"ur 32-Bit-Systeme:\\$;C:\backslash ProgramFiles(x86)\backslash Java\backslash jdk1.8.0_231\backslash bin$\\
Als n\"achstes wird \textit{JAVA\_HOME} gesetzt. Unter \textit{Systemvariablen} wird mit \textit{Neu} ein \"ahnliches Fenster ge\"offnet, bei dem als  \textit{Name der Variable JAVA\_HOME} und als Wert Folgendes gesetzt wird:\newline
F\"ur 64-Bit-Systeme:\\$C:\backslash ProgramFiles\backslash Java\backslash jdk1.8.0_231$
F\"ur 32-Bit-Systeme:\\$C:\backslash ProgramFiles(x86)\backslash Java\backslash jdk1.8.0_231$\\
\subsubsection{LeJOS}
Als N\"achstes folgt die Installation von leJOS, die Schnittstelle zwischen dem PC und dem Roboter. Das Programm kann
\href{https://sourceforge.net/projects/ev3.lejos.p/files/0.9.1-beta/leJOS_EV3_0.9.1-beta_win32_setup.exe/download}{\textbf{hier}\footnote{$https://sourceforge.net/projects/ev3.lejos.p/files/0.9.1-beta/leJOS_EV3_0.9.1-beta_win32_setup.exe/download$}}
heruntergeladen werden und entsprechend der Anweisungen installiert werden.

\subsubsection{Ejdk}
Damit der Roboter den Java-Code verarbeiten kann, muss ein neues Betriebssystem auf der MicroSD-Karte installiert werden. Daf\"ur wird das 
\href{https://download.oracle.com/otn/java/jdk/8u231-b11/5b13a193868b4bf28bcb45c792fce896/jdk-8u231-windows-i586.exe}{\textbf{ejdk}\footnote{$https://download.oracle.com/otn/java/jdk/8u231-b11/5b13a193868b4bf28bcb45c792fce896/jdk-8u231-windows-i586.exe$}} ben\"otigt.

\subsubsection{IntelliJ}
Zur Programmierung wird eine IDE ben\"otigt, wir verwenden dazu in unseren Workshops intelliJ, welche 
\href{https://download.jetbrains.com/idea/ideaIC-2018.3.6.exe?_ga=2.13038195.579836862.1576405054-449566094.1572879017}{\textbf{hier}\footnote{$https://download.jetbrains.com/idea/ideaIC-2018.3.6.exe?_ga=2.13038195.579836862.1576405054-449566094.1572879017$}}
heruntergeladen werden und entsprechend der weiteren Anweisungen installiert werden kann. Zur Kommunikation mit dem Roboter muss ein entsprechendes Plugin installiert werden, dies möglicht über \textit{Datei} \rightarrow{} \textit{Einstellungen} \rightarrow{} \textit{Plugins}. Danach findet man das entsprechende Plugin mit dem Suchbegriff \textit{ev3}, sodass man es \"uber \textit{installieren} dem Programm hinzuf\"ugen kann.

\section{Einrichtung der Roboter}
Um dieses Betriebssystem auf dem Roboter zu installieren, muss das Programm unter \textit{$C:\backslash Program Files\backslash leJOS EV3\backslash bin$} gestartet werden.\newline
Danach muss die eingesteckte microSD-Karte als Laufwerk ausgew\"ahlt werden und das soeben heruntergeladene JRE ausgew\"ahlt werden. Mit \textit{Create} wird dann die Software auf der microSD-Karte installiert und kann in den Roboter gesteckt werden um sie mit einem Start des Roboters zu installieren.

\section{Einrichtung Mindroid}
Dieser Abschnitt ist nur relevant, wenn der Mindroid-Workshop durchgef\"uhrt werden soll.

\subsection{PC}
F\"ur die Kommunikation zum Handy m\"ussen auf dem PC verschiedene Regeln in der Firewall hinzugef\"ugt werden.

\subsection{Handy}
Damit auf den Handys die Mindroid-App installiert werden kann, dazu m\"ussen die Handys gerootet werden, dazu gibt es im repository das entsprechende Skript \textit{}.
\subsection{Router}

\subsection{Roboter}
Auf dem Roboter muss f\"ur die Nutzung ein spezielles Programm gestartet werden, welches vorher \"uber das Skript \textit{script} \"ubertragen werden kann.

\section{Pr\"asentation}
Unter folgendem \href{https://download.jetbrains.com/idea/ideaIC-2018.3.6.exe?_ga=2.13038195.579836862.1576405054-449566094.1572879017}{\textbf{Link}\footnote{$https://download.jetbrains.com/idea/ideaIC-2018.3.6.exe?_ga=2.13038195.579836862.1576405054-449566094.1572879017$}} 
sind die von uns genutzten Pr\"asentationen zu finden, hierbei ist es aber immer sinnvoll, den Inhalt an das Niveau und Alter der Zuh\"orer anzupassen.

%\end{figure}


\cfoot{\textcolor{lightgray} \today}




\end{document}
