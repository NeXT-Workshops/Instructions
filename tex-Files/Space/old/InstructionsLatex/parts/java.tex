\section{Kurze "Ubersicht "uber Java}
	\arrayrulecolor{white}
	\begin{table}[h]
		\renewcommand{\arraystretch}{1.5}
		\begin{center}
			\begin{tabular}{|p{0.3\textwidth}|p{0.7\textwidth}|}
				\hline
				Schleife mit Bedingung& while(Bedingung) \{Programmcode\} \tabularnewline
				\space & \scriptsize Beispiel: while(i<100)\{...\} \tabularnewline
				\hline
				Z"ahlschleife& for(Start; Bedingung; Z"ahlschritte) \{Programmcode\} \tabularnewline
				\space & \scriptsize Beispiel: for(int i=0;i<10;i++)\{...\} \tabularnewline
				\hline
				Bedingung& if(Bedingung) \tabularnewline
				\space& \{wenn die Bedingung wahr ist, wird dieser Code ausgef"uhrt\} \tabularnewline
				\space & else \tabularnewline
				\space & \{wenn die Bedingung falsch ist, wird dieser Code ausgef"uhrt\}
				\tabularnewline
				\hline			
			\end{tabular}
		\end{center}
	\end{table}
	