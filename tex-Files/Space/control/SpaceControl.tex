%%
% !TeX program = lualatex
%%

\documentclass[
    DIV=calc,
    IMRAD=false,
	ngerman,
	accentcolor=1c,% Farbe für Hervorhebungen auf Basis der Deklarationen in den Corporate Design Richtlinien
%	logofile=example-image, %Falls die Logo Dateien nicht vorliegen
	marginpar=false,
	identbarcolor=1c,
	]{tudapub}
\usepackage[T1]{fontenc}
\usepackage[english, main=ngerman]{babel}
\usepackage[babel]{csquotes}

\usepackage{hologo}

\usepackage{floatrow}
%\usepackage{subfig}
\newfloatcommand{capbtabbox}{table}[][\FBwidth+5cm]
\usepackage{blindtext}
\usepackage{colortbl}
\usepackage{ifthen}

%\usepackage[demo]{graphicx}
\usepackage{caption}
\usepackage[skip=0cm,list=true,labelfont=it]{subcaption}

\newcommand{\unit}[1]{{\rm\,#1}}

\newboolean{sensorsDetailed}
\setboolean{sensorsDetailed}{true}


\begin{document}

%Zusätzliche Metadaten für PDF/A. In diesem Fall notwendig, weil Titel ein Makro enthält.
\Metadata{
	author=NeXT,
	title=Space Workshop Dokumentation,
	%subject=Basisdokumentation und Template zur Nutzung der tudapub-Dokumentenkasse,
	%date=2019-04-29,
	%keywords=TU Darmstadt \sep Corporate Design \sep LaTeX
}




\title{Space Workshop\newline Dokumentation}
\subtitle{NeXT Generation on Campus}
%\author{Marei Peischl\thanks{pei\TeX{} \TeX{}nical Solutions}\and der \TeX-Löwe}
\date{}
\titleimage{
%	%Folgende Box kann selbstverständlich durch ein mit \includegraphics geladenes Bild ersetzt werden.
\bigskip
\bigskip
\bigskip
\bigskip
\bigskip
\bigskip
\bigskip
\bigskip
\bigskip
\bigskip
\bigskip
\includegraphics[width=\textwidth]{images/title.jpg}
	%\color{black!30}\rule{\width}{\height}
}


%Varianten der Infoboxen
\addTitleBox{\includegraphics[width=\linewidth]{../../ist_logo.pdf}}
%\addTitleBoxLogo{example-image}
%\addTitleBoxLogo*{\includegraphics[width=.3\linewidth]{example-image}}



\maketitle


\newpage

%\end{figure}
\section{Die EV3-Steuereinheit}
%	\begin{figure}[h]
%		\begin{floatrow}
%			\ffigbox{%
%				\includegraphics[width=9cm]{images/brick.png}%\rule{3cm}{3cm}%
%			}{%
%				\caption{EV3-Brick}%
%			}
%			\capbtabbox{%
%				\footnotesize
%				\begin{tabular}{|c|c|} \hline
%					Motorausg"ange & MotorPort.A \\
%					&MotorPort.B\\
%					&MotorPort.C\\
%					&MotorPort.D\\ \hline
%					Links & LEFT \\ \hline
%					Rechts & RIGHT \\ \hline
%					Oben & UP \\ \hline
%					Unten & DOWN \\ \hline
%					Mitte & ENTER \\ \hline
%					Oben-Links & ESCAPE \\ \hline
%					Sensoreing"ange & SensorPort.S1 \\
%					&SensorPort.S2\\
%					&SensorPort.S3\\
%					&SensorPort.S4\\ \hline
%				\end{tabular}
%			}{%
%				\caption{Tastenbenennung}%
%			}
%		\end{floatrow}
%	\end{figure}

	
	\begin{figure}[hptb]
		\begin{subfigure}{.65\textwidth}
			\includegraphics[width=\textwidth]{images/brick.png}
			\\
			\caption{EV3-Brick}
			\label{fig:g1}
		\end{subfigure}%
		\hfill
		\begin{subfigure}{.35\textwidth}
			%\centering
			\begin{tabular}{|c|c|} \hline
				Motorausg"ange & MotorPort.A \\
				&MotorPort.B\\
				&MotorPort.C\\
				&MotorPort.D\\ \hline
				Links & LEFT \\ \hline
				Rechts & RIGHT \\ \hline
				Oben & UP \\ \hline
				Unten & DOWN \\ \hline
				Mitte & ENTER \\ \hline
				Oben-Links & ESCAPE \\ \hline
				Sensoreing"ange & SensorPort.S1 \\
				&SensorPort.S2\\
				&SensorPort.S3\\
				&SensorPort.S4\\ \hline
			\end{tabular}
			\newline  \newline \newline \newline \newline
			\caption{Tastenbenennung}
			\label{fig:g2}
		\end{subfigure}
		%	\caption{Figure caption.}
	\end{figure}
	
	Durch Dr"ucken der mittleren und unteren Taste wird das laufende Programm beendet.
	

		
\section{Motorsteuerung}
	Dem Roboter stehen 2 starke gro\ss{}e und ein schw"acherer mittelgro\ss{}er Motor zur Verf"ugung.\\ \\
	\begin{figure}[h]
		\begin{floatrow}
			\ffigbox{%
				\includegraphics[width=4cm]{images/bigMotor.png}%\rule{3cm}{3cm}%
			}{%
				\caption{gro\ss{}er Motor}%
			}
			\ffigbox{%
				\includegraphics[width=4cm]{images/mediumMotor.png}%\rule{3cm}{3cm}%
				
			}{%
				\caption{mittlerer Motor}%
			}
		\end{floatrow}
	\end{figure}

	
	
	\begin{table}[h]
		\begin{tabular}{|p{0.2\textwidth}| p{0.7\textwidth}|}
			\hline
			Motorausgang & lejos.hardware.port.MotorPort\\ \hline
			gro\ss{}er Motor& lejos.hardware.motor.EV3LargeRegulatedMotor
			 \\ \hline mittlerer Motor & lejos.hardware.motor.EV3LargeMediumRegulatedMotor\\ \hline 
		\end{tabular}
		\caption{ben"otigte Imports}
	\end{table}
 
	\begin{table}[H]
		\begin{tabular}{|p{0.2\textwidth}| p{0.7\textwidth}|}
			\hline
			forward()& Motor dreht sich vorw"arts \\ \hline 
			backward() &  Motor dreht sich r"uckw"arts\\ \hline 
			stop() & Motor stoppt\\ \hline
			rotate(int a) & Motor dreht sich um a Grad\\ \hline
			setSpeed(int x) & setzt die Geschwindigkeit des Motors \\
			& Das Maximum ist hierbei 800\\ \hline
		\end{tabular}
		\caption{wichtige Methoden}
	\end{table}

	Um die Motoren verwenden zu k"onnen, muss zuerst der Motorport und der entsprechende Motor oben in der Datei importiert werden.\\ 
	\textbf{Bsp.: import lejos.hardware.motor.EV3LargeRegulatedMotor;}\\ \\
	Als n"achstes muss 
	Im folgenden Beispiel steht \glqq name\grqq{} f"ur einen frei w"ahlbaren Namen und \glqq X\grqq{} f"ur den Port, also A, B, C oder D.\\
	\textbf{EV3LargeRegulatedMotor name = new EV3LargeRegulatedMotor(MotorPort.X);} \\ 
	\textbf{Bsp.: EV3LargeRegulatedMotor motor = new EV3LargeRegulatedMotor(MotorPort.A);}\\ \\
	Damit sich der Motor bewegt, m"ussen dem Motor nach folgendem Muster eine der gegebenen Methoden gegeben werden.\\
	\textbf{name.Methode;}\\
	\textbf{Bsp.: motor.forward();}

	
\section{Warten}
	Beim Programmieren ist es immer wieder notwendig, Pausen einzubauen.\\
	Hierf"ur ist der Import \textbf{lejos.utitilty.Delay} notwendig.\\
	Mit der Methode \textbf{Delay.msDelay(int time)} wird eine Pause mit \glqq time \grqq{} Millisekunden ausgef"uhrt. (1000 Milisekunden sind eine Sekunde)\\ \\
	\textbf{Bsp.: Delay.msDelay(2000);}
	
	
\section{Tasten}
	\begin{table}[h]
		\begin{tabular}{|p{0.2\textwidth}| p{0.7\textwidth}|}
			\hline
			Tasten& lejos.hardware.Button
			\\ \hline 
		\end{tabular}
		\caption{ben"otigter Import}
	\end{table}
	
	\begin{table}[H]
		\begin{tabular}{|p{0.2\textwidth}| p{0.7\textwidth}|}
			\hline
			isDown()& wahr, wenn Taste gedr"uckt \\ \hline 
			isUp() &  wahr, wenn Taste nicht gedr"uckt\\ \hline 
			waitForPress() & wartet, bis Taste gedr"uckt\\ \hline
		\end{tabular}
		\caption{wichtige Methoden}
	\end{table}
Die Methoden werden nach folgendem Muster aufgerufen: \textbf{Button.name.Methode} wobei der \glqq name\grqq{} f"ur die Bezeichnung der Taste steht.\\
\textbf{Bsp.: Button.LEFT.waitForPress();}

Mit der Methode \textbf{\glqq Button.LEDPattern(int i)\grqq{}} wird die LED unter den Tasten gesteuert. Hierbei leuchtet die LED mit unterschiedlichem \textbf{i} von null bis acht in verschiedenen Rhythmen und Farben.

\section{Lautsprecher}
	\begin{table}[h]
		\begin{tabular}{|p{0.2\textwidth}| p{0.7\textwidth}|}
			\hline
			Lautsprecher& lejos.hardware.Sound
			\\ \hline 
		\end{tabular}
		\caption{ben"otigter Import}
	\end{table}
	
	\begin{table}[H]
		\begin{tabular}{|p{0.2\textwidth}| p{0.7\textwidth}|}
			\hline
			beep()& spielt einen Ton ab \\ \hline 
			twoBeeps() &  spielt den gleichen Ton zweimal ab\\ \hline 
			beepSequence & spielt eine absteigende Tonfolge ab\\ \hline
			beepSequenceUp()& spielt eine aufsteigende Tonfolge ab \\ \hline 
			buzz()& summt \\ \hline 
			setVolume(int vol)& setzt die Lautst"arke auf den Wert vol (0-100) \\ \hline 
		\end{tabular}
		\caption{wichtige Methoden}
	\end{table}
	Die Methoden werden nach folgendem Muster aufgerufen: \textbf{Sound.Methode}\\
	\textbf{Bsp.: Sound.Beep();}

\section{Display}
	\begin{figure}[h]
		\centering 
		\includegraphics[width=5cm]{images/EV3-CoordSystem.png}
		\caption{Koordinatensystems des Displays}
	\end{figure}
	\begin{table}[h]
		\begin{tabular}{|p{0.2\textwidth}| p{0.7\textwidth}|}
			\hline
			Display& lejos.hardware.lcd.LCD
			\\ \hline 
		\end{tabular}
		\caption{ben"otigter Import}
	\end{table}
	
	\begin{table}[H]
		\begin{tabular}{|p{0.2\textwidth}| p{0.7\textwidth}|}
			\hline
			drawString(String str, int x, int y)& Zeigt einen Text an, beginnend in Spalte x und Zeile y \\ \hline 
			drawInt(int i, int x, int y) &  Zeigt eine Ganzzahl an, beginnend in Spalte x und Zeile y\\ \hline 
			clear() & l"oscht den Inhalt des Displays\\ \hline
			clear(int y)& l"oscht den Inhalt der y-ten Zeile \\ \hline 
			scroll()& verschiebt den Inhalt um eine Zeile nach oben \\ \hline 
		\end{tabular}
		\caption{wichtige Methoden}
	\end{table}

	Das Koordinatensystem auf dem Display hat seinen Ursprung (0,0) oben links und geht in x-Richtung nach rechts 16 Spalten und in y-Richtung nach unten 8 Zeilen. \\ \\
	Das Display wird angesprochen mit: \textbf{LCD.Methode}\\
	\textbf{Bsp.: LCD.drawString(\glqq Hallo Welt!\grqq{}, 0, 0);}

\ifthenelse{\boolean{sensorsDetailed}}
{\section{Sensoren}
%\begin{figure}[H]
%	\ffigbox[\textwidth][]{%
%		\begin{subfloatrow}
%			\ffigbox[\FBwidth][]
%			{\caption{Ultraschallsensor}}
%			{\includegraphics[]{images/ultrasonic.jpg}}
%			\ffigbox[\FBwidth][]
%			{\caption{Farbsensor}}
%			{\includegraphics[]{images/color.png}}
%		\end{subfloatrow}%\hspace*{\columnsep}%
%		\begin{subfloatrow}
%			\ffigbox[\FBwidth][]
%			{\caption{Winkelsensor}}
%			{\includegraphics[]{images/gyro.png}}
%			\ffigbox[\FBwidth][]
%			{\caption{Ber"uhrungssensor}}
%			{\includegraphics[]{images/touch.png}}
%		\end{subfloatrow}
%	}{}
%\end{figure}

\begin{figure}[hptb]
	\begin{subfigure}{.225\textwidth}
		\includegraphics[width=\textwidth]{images/ultrasonic.jpg}
		\\
		\caption{Ultraschallsensor}
		\label{fig:g1}
	\end{subfigure}%
	\hfill
	\begin{subfigure}{.225\textwidth}
		\includegraphics[width=\textwidth]{images/color.png}
		\\
		\caption{Farbsensor}
		\label{fig:g2}
	\end{subfigure}
	\hfill
	\begin{subfigure}{.225\textwidth}
		\includegraphics[width=\textwidth]{images/gyro.png}
		\\
		\caption{Winkelsensor}
		\label{fig:g2}
	\end{subfigure}
	\hfill
	\begin{subfigure}{.225\textwidth}
		\includegraphics[width=\textwidth]{images/touch.png}
		\\
		\caption{Ber"uhrungssensor}
		\label{fig:g2}
	\end{subfigure}
%	\caption{Figure caption.}
\end{figure}

\begin{table}[h]
	\begin{tabular}{|p{0.2\textwidth}| p{0.7\textwidth}|}
		\hline
		Sensorausgang & lejos.hardware.port.SensorPort\\ \hline
		Ultraschallsensor& lejos.hardware.sensor.EV3UltrasonicSensor
		\\ \hline 
		Farbsensor & lejos.hardware.sensor.EV3ColorSensor\\ \hline 
		Winkelsensor & lejos.hardware.sensor.EV3GyroSensor\\ \hline 
		Ber"uhrungssensor & lejos.hardware.sensor.EV3TouchSensor\\ \hline 
	\end{tabular}
	\caption{ben"otigte Imports}
\end{table}

\begin{table}[H]
	\begin{tabular}{|p{0.2\textwidth}| p{0.7\textwidth}|}
		\hline
		setCurrentMode\newline (String mode)& setzt den Modus des Sensors \\ \hline 
		sampleSize() &  gibt die Anzahl der zur"uckgegeben Werte zurück\\ \hline 
		fetchsample(float[] signal, int offset) & Sensor misst und speichert es im Array \textbf{signal} ab Stelle \textbf{offset}\\ \hline
		getColorID() &  Farbsensor gibt erkannte Farbe zur"uck (-1=keine Farbe, 0=Rot, 1=Gr"un, 2=Blau, 3=Gelb, 6=Wei\ss{}, 7=Schwarz, 13=Braun)\\ \hline 
	\end{tabular}
	\caption{wichtige Methoden}
\end{table}

\begin{table}[H]
	\begin{tabular}{|p{0.2\textwidth}| p{0.7\textwidth}|}
		\hline
		Ultraschallsensor& Distance (Sensor gibt Distanz in Metern zur"uck) \\ \hline 
		Farbsensor &  Ambient (Sensor gibt Umgebungshelligkeit in Werten zwischen 0 und 1 zur"uck)\\ \hline 
		Winkelsensor& (Sensor gibt Winkel zur"uck)\\ \hline
		Ber"uhrungssensor& (Sensor gibt zur"uck, ob Taste gedr"uckt (1) oder nicht gedr"uckt (0) ist)\\ \hline
	\end{tabular}
	\caption{Modi}
\end{table}

Die Benutzung der Sensoren ist auf den ersten Blick etwas kompliziert, jedoch folgt die Benutzung einem festen Aufbau.\\

Zuerst muss wie bei den Motoren der Sensor benannt werden.\newline
\textbf{Bsp.: EV3UltrasonicSensor ultra = new EV3UltrasonicSensor(SensorPort.S1)}\\ \\
Als n"achstes muss der Sensor in den richtigen Modus gesetzt werden mit \newline \textbf{sensorname.setCurrentMode(String mode)}\\
\textbf{Bsp.: ultra.setCurrentMode(\glqq Distance\grqq{});}
\\ \\
Der n"achste Schritt ist das Anlegen eines Arrays, in dem die Daten gespeichert werden. Hierbei wird direkt mit \textbf{sampleSize()} die Gr"o\ss{}e gesetzt.
\textbf{Bsp.: float[] signal = new float[sensorname.sampleSize()];}
\\ \\
Um neue Daten zu erfassen wird mit dem Sensor die Methode \textbf{fetchSample} aufgerufen und im vorher angelegten Array gespeichert.\\
\textbf{Bsp.: ultra.fetchSample(signal, 0);}

}
{\input{parts/sensorsAPI}}
\section{Kurze "Ubersicht "uber Java}
	\arrayrulecolor{white}
	\begin{table}[H]
		\renewcommand{\arraystretch}{1.5}
		\begin{center}
			\begin{tabular}{|p{0.3\textwidth}|p{0.7\textwidth}|}
				\hline
				Schleife mit Bedingung& while(Bedingung) \{Programmcode\} \tabularnewline
				\space & \scriptsize Beispiel: while(i<100)\{...\} \tabularnewline
				\hline
				Z"ahlschleife& for(Start; Bedingung; Z"ahlschritte) \{Programmcode\} \tabularnewline
				\space & \scriptsize Beispiel: for(int i=0;i<10;i++)\{...\} \tabularnewline
				\hline
				Bedingung& if(Bedingung) \tabularnewline
				\space& \{wenn die Bedingung wahr ist, wird dieser Code ausgef"uhrt\} \tabularnewline
				\space & else \tabularnewline
				\space & \{wenn die Bedingung falsch ist, wird dieser Code ausgef"uhrt\}
				\tabularnewline
				\hline			
			\end{tabular}
		\end{center}
	\end{table}

\subsection{Exkurs Arrays}
Ein Array kann man sich als Schrank mit verschiedenen Schubladen vorstellen, mit den sogenannten Indizes kann man auf die verschiedenen Stellen im Array zugreifen. \newline
Die Erstellung des Arrays ist \"ahnlich wie bei normalen Variablen. Hierbei wird erst der Datentyp mit eckigen Klammern, dann der Variablenname geschrieben. Als n\"achstes muss die Gr\"o\ss{}e n des Arrays festgelegt. \textbf{Bsp.: int[] array = new int [n];} \newline
Nun kann auf die die einzelnen Elemente mit dem Index i zugegriffen werden. \textbf{Bsp.: array[i]}
	
\section{Quellen}
	Die Dokumentation mit weiteren Methoden befindet sich unter:\\
	lejos.org/ev3/docs/\\ \\
	
	Die Lejos Software kann auf der folgenden Seite heruntergeladen werden. Weiterhin gibt es dort ausf"uhrliche Anleitungen zum Einrichten auf dem EV3-Roboter:\\
	lejos.sourceforge.io\\ \\
	
	Der genutzte Java-Editor ist kostenfrei im Netz verf"ugbar:\\
	javaeditor.org\\ \\
	
	Bei Fragen stehen wir immer gerne zur Verf"ugung:\\
	next-generation@etit.tu-darmstadt.de \\ \\
	
	Die Bilder sind der Internetseite des offiziellen Lego-Onlineshops lego.com entnommen. Die Urheberrechte befinden sich im Besitz der LEGO Gruppe, diese Anleitung ist unabh"angig und wurde von der LEGO Gruppe weder autorisiert noch gesponsert.
\cfoot{\textcolor{lightgray} \today}




\end{document}
