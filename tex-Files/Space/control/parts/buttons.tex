\section{Tasten}
	\begin{table}[h]
		\begin{tabular}{|p{0.2\textwidth}| p{0.7\textwidth}|}
			\hline
			Tasten& lejos.hardware.Button
			\\ \hline 
		\end{tabular}
		\caption{ben"otigter Import}
	\end{table}
	
	\begin{table}[H]
		\begin{tabular}{|p{0.2\textwidth}| p{0.7\textwidth}|}
			\hline
			isDown()& wahr, wenn Taste gedr"uckt \\ \hline 
			isUp() &  wahr, wenn Taste nicht gedr"uckt\\ \hline 
			waitForPress() & wartet, bis Taste gedr"uckt\\ \hline
		\end{tabular}
		\caption{wichtige Methoden}
	\end{table}
Die Methoden werden nach folgendem Muster aufgerufen: \textbf{Button.name.Methode} wobei der \glqq name\grqq{} f"ur die Bezeichnung der Taste steht.\\
\textbf{Bsp.: Button.LEFT.waitForPress();}

Mit der Methode \textbf{\glqq Button.LEDPattern(int i)\grqq{}} wird die LED unter den Tasten gesteuert. Hierbei leuchtet die LED mit unterschiedlichem \textbf{i} von null bis acht in verschiedenen Rhythmen und Farben.
