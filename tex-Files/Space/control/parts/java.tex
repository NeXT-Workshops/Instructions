\section{Kurze "Ubersicht "uber Java}
	\arrayrulecolor{white}
	\begin{table}[H]
		\renewcommand{\arraystretch}{1.5}
		\begin{center}
			\begin{tabular}{|p{0.3\textwidth}|p{0.7\textwidth}|}
				\hline
				Schleife mit Bedingung& while(Bedingung) \{Programmcode\} \tabularnewline
				\space & \scriptsize Beispiel: while(i<100)\{...\} \tabularnewline
				\hline
				Z"ahlschleife& for(Start; Bedingung; Z"ahlschritte) \{Programmcode\} \tabularnewline
				\space & \scriptsize Beispiel: for(int i=0;i<10;i++)\{...\} \tabularnewline
				\hline
				Bedingung& if(Bedingung) \tabularnewline
				\space& \{wenn die Bedingung wahr ist, wird dieser Code ausgef"uhrt\} \tabularnewline
				\space & else \tabularnewline
				\space & \{wenn die Bedingung falsch ist, wird dieser Code ausgef"uhrt\}
				\tabularnewline
				\hline			
			\end{tabular}
		\end{center}
	\end{table}

\subsection{Exkurs Arrays}
Ein Array kann man sich als Schrank mit verschiedenen Schubladen vorstellen, mit den sogenannten Indizes kann man auf die verschiedenen Stellen im Array zugreifen. \newline
Die Erstellung des Arrays ist \"ahnlich wie bei normalen Variablen. Hierbei wird erst der Datentyp mit eckigen Klammern, dann der Variablenname geschrieben. Als n\"achstes muss die Gr\"o\ss{}e n des Arrays festgelegt. \textbf{Bsp.: int[] array = new int [n];} \newline
Nun kann auf die die einzelnen Elemente mit dem Index i zugegriffen werden. \textbf{Bsp.: array[i]}
	